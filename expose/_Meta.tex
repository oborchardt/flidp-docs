% Informationen ------------------------------------------------------------
% 	Definition von globalen Parametern, die im gesamten Dokument verwendet
% 	werden können (z.B auf dem Deckblatt etc.).
% --------------------------------------------------------------------------
\newcommand{\titel}{Exposé\\Federated Learning With Individualized Differential Privacy}
\newcommand{\art}{Master's Thesis} %Bachelorarbeit
\newcommand{\ort}{Leipzig}
\newcommand{\hochschule}{Leipzig University}
\newcommand{\fachgebiet}{Database Group}
\newcommand{\fakultaet}{Faculty of Mathematics and Computer Science}
\newcommand{\institut}{Department of Computer Science}
\newcommand{\autor}{Ole Borchardt}
\newcommand{\matrikelnr}{3725924}
\newcommand{\erstbetreuer}{Prof. Dr. Erhard Rahm}
\newcommand{\zweitbetreuer}{XXXXX}
\newcommand{\jahr}{2024}
\newcommand{\invnr}{1337}
\newcommand{\eingereicht}{xx.xx.xxxx}

% Eigene Befehle
\newcommand{\todo}[1]{\textbf{\textsc{\textcolor{red}{(TODO: #1)}}}}

% Autorennamen in small caps
\newcommand{\AutorZ}[1]{\textsc{#1}}
\newcommand{\Autor}[1]{\AutorZ{\citeauthor{#1}}}

% Befehle zur semantischen Auszeichnung von Text
\newcommand{\NeuerBegriff}[1]{\textbf{#1}}
\newcommand{\Fachbegriff}[1]{\textit{#1}}
\newcommand{\Prozess}[1]{\textit{#1}}
\newcommand{\Webservice}[1]{\textit{#1}}
\newcommand{\Eingabe}[1]{\texttt{#1}}
\newcommand{\Code}[1]{\texttt{#1}}
\newcommand{\Datei}[1]{\texttt{#1}}
\newcommand{\Datentyp}[1]{\textsf{#1}}
\newcommand{\XMLElement}[1]{\textsf{#1}}

% Abkürzungen
\newcommand{\vgl}{Vgl.\ }
\newcommand{\ua}{\mbox{u.\,a.\ }}
\newcommand{\zB}{\mbox{z.\,B.\ }}
\newcommand{\bs}{$\backslash$}

% Einfache Anführungszeichen in texttt
\newcommand{\sq}{\textquotesingle}

