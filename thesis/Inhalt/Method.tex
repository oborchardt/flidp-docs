\chapter{Experiments}

\section{Datasets and models}

In dieser Arbeit wird der vorgeschlagene Algorithmus empirisch auf synthetischen und realistischen Datensätzen evaluieren. Dabei orientiere ich mich an \textcite{aldaghri:2023}. In ihrer Arbeit haben sie realistische Datensätze von Tensorflow Federated genutzt, nämlich MNIST und EMNIST. Darüber hinaus haben sie synthetische Datensätze basierend auf MNIST erstellt. In diesen haben sie die Bilder nach Ziffern gruppiert und die Daten für die einzelnen Clients jeweils aus einer dieser Gruppen gesamplet. Darüber hinaus wurden einzelne Ziffern nur privaten Clients zugewiesen und die Auswirkungen dieser Setups ausgewertet.

Wie \textcite{boenisch:2023} lege ich Gruppen von Clients mit unterschiedlichen Privacy Budgets $\epsilon$ zugrunde. Dabei wähle ich die Budgets und Gruppengrößen wie sie: $\epsilon_p = \{1,2,3\}$ und $(34\%, 43\%, 23\%)$ bzw. $(54\%, 37\%, 9\%)$. Darüber hinaus evaluiere ich das Training mit dem strengsten Budget für alle Clients und mit dem FedAvg ohne Privacy Garantien, um sinnvolle Baselines für meinen Algorithmus zu erhalten.
