% Anpassung des Seitenlayouts ----------------------------------------------
% 	siehe Seitenstil.tex
% --------------------------------------------------------------------------
\usepackage[
	automark,			% Kapitelangaben in Kopfzeile automatisch erstellen
	headsepline,	% Trennlinie unter Kopfzeile
	ilines				% Trennlinie linksbündig ausrichten
]{scrlayer-scrpage}
\usepackage{scrhack} % Disable some warnings

\usepackage{pseudocode}
\usepackage{nicefrac}

% Für eine schöne Anordnung von Bildern
%\usepackage{subfigure}

\usepackage{dsfont}
%\usepackage{color}
%
%% Define user colors using the RGB model
%\definecolor{yellow}{rgb}{0.0,1.0,0.0}
%\definecolor{rot}{rgb}{1.0,0.0,0.0}

% Anpassung an Landessprache -----------------------------------------------
% 	Verwendet globale Option german siehe \documentclass
% --------------------------------------------------------------------------
\usepackage[ngerman]{babel}

% Umlaute ------------------------------------------------------------------
% 		Umlaute/Sonderzeichen wie äöüß direkt im Quelltext verwenden (CodePage).
%		Erlaubt automatische Trennung von Worten mit Umlauten.
% --------------------------------------------------------------------------
\usepackage[utf8]{inputenc}
%\usepackage[T1]{fontenc}
%\usepackage{ae} % "schöneres" ä
\usepackage{textcomp} % Euro-Zeichen etc.
\usepackage{lmodern} % schööön

% Grafiken -----------------------------------------------------------------
% 		Einbinden von Grafiken [draft oder final]
% 		Option [draft] bindet Bilder nicht ein - auch globale Option
% --------------------------------------------------------------------------
\usepackage[dvips,final]{graphicx}
\usepackage{wrapfig}
\graphicspath{{Bilder/}} % Dort liegen die Bilder des Dokuments

% Befehle aus AMSTeX für mathematische Symbole z.B. \boldsymbol \mathbb ----
\usepackage{amsmath,amsfonts,amsthm}

% Für Index-Ausgabe; \printindex -------------------------------------------
\usepackage{makeidx}

% Einfache Definition der Zeilenabstände und Seitenränder etc. -------------
\usepackage{setspace}
\usepackage{geometry}

% für gedrehte Tabellen
\usepackage{rotating} 

% Symbolverzeichnis --------------------------------------------------------
% 	Symbolverzeichnisse bequem erstellen, beruht auf MakeIndex.
% 		makeindex.exe %Name%.nlo -s nomencl.ist -o %Name%.nls
% 	erzeugt dann das Verzeichnis. Dieser Befehl kann z.B. im TeXnicCenter
%		als Postprozessor eingetragen werden, damit er nicht ständig manuell
%		ausgeführt werden muss.
%		Die Definitionen sind ausgegliedert in die Datei Abkuerzungen.tex.
% --------------------------------------------------------------------------
\usepackage[intoc]{nomencl}
  \let\abbrev\nomenclature
  \renewcommand{\nomname}{Abkürzungsverzeichnis}
  \setlength{\nomlabelwidth}{.25\hsize}
  \renewcommand{\nomlabel}[1]{#1 \dotfill}
  \setlength{\nomitemsep}{-\parsep}

% Zum Umfließen von Bildern -------------------------------------------------
\usepackage{floatflt}

% Zum Einbinden von Programmcode --------------------------------------------
\usepackage{listings}
\usepackage{xcolor} 
\definecolor{hellgelb}{rgb}{1,1,0.9}
\definecolor{colKeys}{rgb}{0,0,1}
\definecolor{colIdentifier}{rgb}{0,0,0}
\definecolor{colComments}{rgb}{1,0,0}
\definecolor{colString}{rgb}{0,0.5,0}
\lstset{%
    float=hbp,%
    basicstyle=\texttt\small, %
    identifierstyle=\color{colIdentifier}, %
    keywordstyle=\color{colKeys}, %
    stringstyle=\color{colString}, %
    commentstyle=\color{colComments}, %
    columns=flexible, %
    tabsize=2, %
    frame=single, %
    extendedchars=true, %
    showspaces=false, %
    showstringspaces=false, %
    numbers=left, %
    numberstyle=\tiny, %
    breaklines=true, %
    backgroundcolor=\color{hellgelb}, %
    breakautoindent=true, %
%    captionpos=b%
}

% Lange URLs umbrechen etc. -------------------------------------------------
\usepackage{url}


%% Wichtig für korrekte Zitierweise ------------------------------------------

\usepackage[autocite=inline, sorting=none, backend=biber]{biblatex}
\addbibresource{quellen.bib} % Name der .bib-Datei

\usepackage{csquotes} % Empfohlen, um Zitierten Text richtig darzustellen

% ermöglicht Zeilenumbrüche in Captions
\usepackage{caption}


% PDF-Optionen --------------------------------------------------------------
\usepackage[
bookmarks,
bookmarksopen=true,
pdftitle={\titel},
pdfauthor={\autor},
pdfcreator={\autor},
pdfsubject={\titel},
pdfkeywords={\titel},
colorlinks=true,
%linkcolor=red, % einfache interne Verknüpfungen
%anchorcolor=black,% Ankertext
%citecolor=blue, % Verweise auf Literaturverzeichniseinträge im Text
%filecolor=magenta, % Verknüpfungen, die lokale Dateien öffnen
%menucolor=red, % Acrobat-Menüpunkte
%urlcolor=cyan, 
% für die Druckversion können die Farben ausgeschaltet werden:
linkcolor=black, % einfache interne Verknüpfungen
anchorcolor=black,% Ankertext
citecolor=black, % Verweise auf Literaturverzeichniseinträge im Text5
filecolor=black, % Verknüpfungen, die lokale Dateien öffnen
menucolor=black, % Acrobat-Menüpunkte
urlcolor=black, 
%backref,
%pagebackref,
plainpages=false,% zur korrekten Erstellung der Bookmarks
pdfpagelabels,% zur korrekten Erstellung der Bookmarks
hypertexnames=false,% zur korrekten Erstellung der Bookmarks
linktocpage % Seitenzahlen anstatt Text im Inhaltsverzeichnis verlinken
]{hyperref}

% Zum fortlaufenden Durchnummerieren der Fußnoten ---------------------------
\usepackage{chngcntr}


% für lange Tabellen
\usepackage{longtable}
\usepackage{array}
\usepackage{ragged2e}
\usepackage{lscape}

\usepackage{supertabular}

% Spaltendefinition rechtsbündig mit definierter Breite ---------------------
\newcolumntype{w}[1]{>{\raggedleft\hspace{0pt}}p{#1}}

% Formatierung von Listen ändern
\usepackage{paralist}
% Standardeinstellungen:
% \setdefaultleftmargin{2.5em}{2.2em}{1.87em}{1.7em}{1em}{1em}

\usepackage{tablefootnote}
% für Ausblenden der Seitenzahl
\usepackage{lipsum}

% für subfigures
\usepackage{caption}
\usepackage{subcaption}

% für Durchschnittszeichen
\usepackage{wasysym}

% für tabellen
\usepackage{multirow}

% für Algorithmen
\usepackage{algorithm2e}
\RestyleAlgo{ruled}

% für schönere Tabellen
\usepackage{booktabs}